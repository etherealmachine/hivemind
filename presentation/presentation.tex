\documentclass{beamer}
\usetheme{Boadilla}
\usepackage{havannah}

\title[Learning MCTS]{Evolutionary Learning of Policies for MCTS Simulations}
\author[Pettit, Helmbold]{James Pettit, David Helmbold}
\institute[UCSC]{
  University of California, Santa Cruz\\
  \texttt{jpettit@soe.ucsc.edu}
}
\date[July 2012]{July 2012}

\begin{document}

\begin{frame}[plain]
  \titlepage
\end{frame}

\begin{frame}{Overview}
\end{frame}

\begin{frame}{The Game of Hex}
% example board, play a few moves, show final position
% useful for experimentation:
% 1. Easy to program
% 2. Clear-cut winning condition (not so for Go)
% 3. Large problem space
	\begin{figure}[tb]
	\resizebox{3.3in}{!}{ \begin{HexBoard}[board size=7]
		  \HGame{d4}
		\end{HexBoard}
		}
	\end{figure}
\end{frame}
\begin{frame}{The Game of Hex}
	\begin{figure}[tb]
	\resizebox{3.3in}{!}{ \begin{HexBoard}[board size=7]
		  \HGame{d4,d3}
		\end{HexBoard}
		}
	\end{figure}
\end{frame}
\begin{frame}{The Game of Hex}
	\begin{figure}[tb]
	\resizebox{3.3in}{!}{ \begin{HexBoard}[board size=7]
		  \HGame{d4,d3,e3}
		\end{HexBoard}
		}
	\end{figure}
\end{frame}
\begin{frame}{The Game of Hex}
	\begin{figure}[tb]
	\resizebox{3.3in}{!}{ \begin{HexBoard}[board size=7]
		  \HGame{d4,d3,e3,f1}
		\end{HexBoard}
		}
	\end{figure}
\end{frame}
\begin{frame}{The Game of Hex}
	\begin{figure}[tb]
	\resizebox{3.3in}{!}{ \begin{HexBoard}[board size=7]
		  \HGame{d4,d3,e3,f1,c2}
		\end{HexBoard}
		}
	\end{figure}
\end{frame}
\begin{frame}{The Game of Hex}
	\begin{figure}[tb]
	\resizebox{3.3in}{!}{ \begin{HexBoard}[board size=7]
		  \HGame{d4,d3,e3,f1,c2,d6,b5,e2,c4,c3,a4,b6,c6,c5,d5,b7,c7,b4,a5,b3,a3,b1,b2,c1,d1}
		\end{HexBoard}
		}
	\end{figure}
\end{frame}

\begin{frame}{MCTS - Overview}
% Minimax game tree cannot be fully constructed
% Use random playouts to estimate minimax value
% Large enough playouts will converge to true minimax value
\end{frame}

\begin{frame}{MCTS - Playout Policy}
% One of the ways to improve the quality of the estimation
\end{frame}

\begin{frame}{Evolutionary Learning}
\end{frame}

\begin{frame}{Encoding}
\end{frame}

\begin{frame}{Results}
\end{frame}

\begin{frame}{Future Work}
\end{frame}

\end{document}
